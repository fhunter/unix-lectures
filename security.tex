\chapter{Безопасность и всё с ней связанное}
\section{Пользователи и группы}
\section{Права доступа к файлам}
Каждый файл, а так же любой объект файловой системы в UNIX имеет своего владельца и группу.

В классической схеме управления правами в UNIX, все права делятся на 3 группы: права владельца объекта, права группы объекта и права  всех остальных пользователей.

Права доступа записываются либо как восьмеричное число вида: 0755, где первая цифра отвечает за права владельца, вторая цифра отвечает за права группы, а последняя цифра за права остальных пользователей. Каждая цифра собирается по битам:

\begin{tabular}{|c|c|}
4& чтение \\
2& запись \\
1& исполнение \\
\end{tabular}

Второй формат записи выглядит так: rwxr-xr-x. Он аналогичен первому. Первая тройка rwx - отвечает за права доступа владельца файла, вторая за права доступа для группы файла и третья для 
остальных пользователей.

\begin{tabular}{|c|c|}
r&чтение \\
w&запись \\
x&исполнение \\
\end{tabular}

Флаг исполнения для каталога означает возможность перехода в этот каталог, а флаг чтения позволяет просмотреть его содержимое. Таким образом можно создать каталог, в который нельзя войти, но можно просмотреть его содержимое.\footnote{Проверить утверждение и дать примеры}
\section{Управление правами доступа}
Для управления правами доступа и владельцем/группой файлов используются команды: chmod, chown, chgrp и umask.

Команда chmod используется для управления правами доступа файлов.

Команда chown позволяет изменить владельца (и опционально группу файла, но для этого требуются права суперпользователя).

Команда chgrp позволяет сменить группу файла не меняя его владельца, и она так же требует прав суперпользователя.
\section{Set UID, Set GID и другие служебные биты}
\section{Почему нельзя работать под root}
\section{Делегирование прав}
\subsection{sudo}
Команда sudo используется для выдачи того, чтобы выдать одному пользователю права другого для выполнения всех команд или какой либо одной.

Для настройки sudo редактируется файл /etc/sudoers или, в Debian-подобных системах, так же возможно создание отдельных файлов в каталоге /etc/sudoers.d/.

Для предотвращения ошибок при правке, желательно использовать команду visudo, которая выполняет проверку синтаксической корректности файла sudoers. Это необходимо во избежании блокировки доступа в систему.

Синтаксис вызова команды sudo можно посмотреть в базе man.
\subsection{su}
Команда su используется для смены идентификатора пользователя или для того, чтобы стать суперпользователем.
\footnote{Дописать}