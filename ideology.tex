\chapter {Идеология unix}
\section {Компоненты}
\subsection{ядро ОС}
\subsection{средства инициализации системы}
\subsection{системные утилиты, обеспечивающие исполнение ядром его функций}
\subsection{средства "боевого обеспечения" - минимальный набор пользовательских утилит}
\subsection{средства "тылового обеспечения" - системные библиотеки}
\subsection{командная оболочка}
\subsection{система документации}
\section {Основные принципы}
\subsection{всё есть поток байтов}
\subsection{лень читать вывод программы - дай это другой программе или конвеер}
\subsection{модульность - программа должна делать одно дело, но хорошо}
\subsection{администратор может всё. Пользователь может только убить себя об стену (вынести содержимое своего \$home) на этом в общем случае его привилегии и заканчиваются.}
\subsection{Права доступа}
\subsection{Типы файлов}
\subsubsection{Файл}
\subsubsection{Каталог}
\subsubsection{Символьное устройство}
\subsubsection{Блочное устройство}
\subsubsection{Мягкая ссылка}
\subsubsection{Жесткая ссылка}
\subsubsection{Именованный канал (FIFO)}
\subsection{Имена файлов - как можно и как нельзя. И главное - как желательно}
\subsection{Понятие потоков stdin, stdout, stderr.}
