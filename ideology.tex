\chapter {Идеология unix}
\section {Компоненты}
\subsection{ядро ОС}
Ядро операционной системы обеспечивает уровень абстракции над аппаратурой компьютера и обеспечивает удобства пользователя.
\subsection{средства инициализации системы}
Чтобы система была готова к работе с пользователем, нужно выполнить определённый набор действий.
Как минимум нужно отконфигурировать сетевую подсистему, смонтировать файловые системы, установить системные часы, запустить необходимые фоновые процессы (например планировщик).
\par
Всё это делается средствами инициализации системы. В общем случае UNIX система при старте запускает исполняемый файл, лежащий в /sbin/init. Далее этот исполняемый файл становится процессом 1 и отслеживает процесс инициализации системы. В дальнейшем я описываю стартовый процесс для SysV системы. Для BSD систем он несколько отличается, но начинается всё там же - в /sbin/init
\par
Следующие шаги инициализации описаны в файле /etc/inittab, который читается процессом init. В нём указана последовательность действий при инициализации, записаны процессы, которые должны стартовать на каждом runlevel-е. По этим правилам, управление передаётся набору скриптов в /etc/rc[0-6].d/, которые отвечают за старт/остановку системных сервисов.
\par
В linux-системах, эти скрипты именуются по маске [S|K][0-9][0-9]имя, и являются символическими ссылками на скрипты находящиеся в /etc/init.d/. Разберём строение имени символической ссылки: S - стартовать сервис на данном runlevel-е.
K - остановить.
цифры - порядок выполнения данных действий.
имя - символическое имя, чтобы сисадмин оставался белым и пушистым, и чтобы волосы у него не седели от стресса.
\subsection{системные утилиты, обеспечивающие исполнение ядром его функций}
Сюда входят утилиты загрузки модулей ядра, демоны создания файлов устройств, утилиты конфигурирования сетевой подсистемы.
\subsection{пользовательских утилит}
\subsection{системные библиотеки}
\subsection{командная оболочка}
Собственно командная оболочка выделена только потому что, это один из основных способов взаимодействия с UNIX системой (графическую подсистему мы пока не рассматриваем). Кроме того, эта же оболочка используется для исполнения скриптов, являющихся основой системы инициализации UNIX.
\subsection{система документации}
Так как система без документации бесполезна, а бумажные руководства пользователь не читает, пока его не клюнет жареный петух, то возникает необходимость в системе документации. Тем более, что пользователь просто не может помнить абсолютно все возможные опции команд и форматы конфигурационных файлов.
\section {Основные принципы}
\subsection{всё есть поток байтов}
\subsection{лень читать вывод программы - дай это другой программе или конвеер}
\subsection{модульность - программа должна делать одно дело, но хорошо}
\subsection{администратор может всё. Пользователь может только убить себя об стену (вынести содержимое своего \$home) на этом в общем случае его привилегии и заканчиваются.}
\subsection{Права доступа}
\subsection{Типы файлов}
\subsubsection{Файл}
\subsubsection{Каталог}
\subsubsection{Символьное устройство}
\subsubsection{Блочное устройство}
\subsubsection{Мягкая ссылка}
\subsubsection{Жесткая ссылка}
\subsubsection{Именованный канал (FIFO)}
\subsection{Имена файлов - как можно и как нельзя. И главное - как желательно}
\subsection{Понятие потоков stdin, stdout, stderr.}
